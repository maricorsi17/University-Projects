L'elaborazione dei dati a disposizione non è da ritenersi conclusa con i metodi applicati in questo lavoro. Numerose sono infatti le alternative ai metodi che si è scelto di utilizzare e potrebbe essere interessante confrontare i risultati ottenuti con quelli di procedure diverse. 

Innanzitutto, per avvalorare le conclusioni è cui si è giunti, sarebbe opportuno continuare a monitorare gli impianti di depurazione in esame in modo tale da poter applicare le stesse analisi a un numero maggiore di dati. 

Si consiglia di approfondire le analisi statistiche, integrandole con test delle ipotesi o test di significatività, al fine di determinare il livello di attendibilità dei risultati ottenuti. Questo approccio presenta lo svantaggio di complicare l'analisi poiché richiede il soddisfacimento di ipotesi aggiuntive e numerose elaborazioni numeriche. Tuttavia, il vantaggio che se ne ricava è quello di avere maggiore confidenza nell'interpretazione del risultato finale.

Infine, potrebbe essere interessante valutare l'applicazione di altri metodi statistici per l'investigazione dei dati, sia per confrontarli con quelli utilizzati che per ricavare informazioni aggiuntive. Alcuni esempi sono:
\begin{itemize}
	\item Analisi delle Componenti Principali (PCA - \textit{Principal Components Analysis}): metodo multivariato per l'individuazione della correlazione tra le variabili;
	\item Analisi di Fourier delle \textit{time series}: calcolo, rappresentazione ed elaborazione di un periodogramma per l'individuazione della periodicità;
	\item Modelli AR (\textit{AutoRegressive}), ARMA (\textit{AutoRegressive Moving Average}) e ARIMA (\textit{AutoRegressive Integrated Moving Average}): modellazione della \textit{time series} nell'ottica di formulare previsioni sul comportamento futuro del sistema.
\end{itemize}

  