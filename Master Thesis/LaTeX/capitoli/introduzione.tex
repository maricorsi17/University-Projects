Lo studio delle condizioni di funzionamento degli impianti di depurazione permette di valutare se le diverse fasi del processo di depurazione si svolgono come previsto oppure se ci sono problematiche da individuare e risolvere. Inoltre, può essere utile anche per stabilire se ci sono dei comparti da potenziare in via preventiva, prima dell'insorgere di eventuali problemi.

Il monitoraggio degli impianti, anche grazie all'utilizzo di misuratori e sonde dotati di \textit{data logger}, mette a disposizione una buona quantità di dati da analizzare. Convenzionalmente questi dati sono elaborati rappresentandoli graficamente e traendo conclusioni dall'analisi visuale dei grafici. Questo metodo consente di ottenere dei risultati validi che, però, dipendono fortemente dall'esperienza e dalla sensibilità dell'analista e che mancano quindi di una base oggettiva.

In questo lavoro di tesi, ai metodi convenzionali, che prevedono lo studio della rappresentazione grafica dei dati grezzi, si sono affiancati dei metodi statistici ampiamente applicati in altri settori, al fine di confermare, smentire o semplicemente arricchire quanto ricavato dai primi. L'analisi visuale dei dati non deve essere però soppiantata da questi altri metodi perché alcune considerazioni non si possono evincere se non per mezzo di una visione di insieme dei dati unitamente all'esperienza dell'osservatore.

I metodi sopracitati sono stati applicati ai dati provenienti da due impianti di depurazione a fanghi attivi situati nella regione Veneto, per il periodo che va dall'inizio dell'anno 2015 alla fine del primo semestre del 2018.
In particolare, dopo aver descritto gli impianti (\autoref{ch:1}) e i dati forniti dal gestore (\autoref{ch:2}), sono stati riportati e interpretati i risultati delle analisi (\autoref{ch:5}). I concetti teorici di cui ci si è serviti sono richiamati nel \autoref{ch:3} per quanto riguarda l'ingegneria sanitaria-ambientale e nel \autoref{ch:4} per la statistica. Infine, nel \autoref{ch:6}, sono elencate alcune idee di elaborazioni da approfondire in lavori futuri.