In questo lavoro si sono studiate le condizioni di funzionamento di due impianti di depurazione a fanghi attivi per mezzo di indagini statistiche, sia convenzionali che più avanzate, condotte sui dati gestionali misurati all'interno degli impianti stessi.
Secondo l'approccio convenzionale, i dati sono stati interpretati per mezzo dell'analisi esplorativa, ovvero ricavando informazioni dalla loro rappresentazione grezza, e tramite alcuni parametri della statistica descrittiva, quali media e percentili. In aggiunta, si sono applicati dei metodi statistici per verificare la correttezza e l'oggettività delle conclusioni a cui si è giunti con i metodi convenzionali.

Per quanto riguarda l'impianto A, nelle condizioni invernali, che sono le più critiche a causa della bassa temperatura, la portata mediana è di 2.500 m\textsuperscript{3}/d, corrispondente a circa 20.000 AE, ed è di molto inferiore alla potenzialità autorizzata.
Il liquame è fortemente concentrato con valori di concentrazione di COD compresi tra 500 e 1.000 mg/L e concentrazioni di azoto comprese tra 60 e 90 mg/L. Le caratteristiche del liquame sono sbilanciate verso i nutrienti e in particolare verso il fosforo. In generale le concentrazioni in ingresso mostrano una lieve tendenza alla diminuzione, mentre le concentrazioni in uscita rispettano sempre i limiti allo scarico. 
Nonostante la portata trattata sia notevolmente inferiore alla potenzialità autorizzata e nonostante l'età del fango e il carico del fango siano tali da garantire un'adeguata nitrificazione, nei mesi invernali si osserva un calo del rendimento di nitrificazione. La conseguenza di ciò è un incremento della concentrazione di azoto ammoniacale in uscita che, oltre a richiedere del tempo per tornare ad assumere i valori usuali, potrebbe portare al mancato rispetto del limite dell'azoto totale qualora non si raggiungesse la soglia minima del 75\% relativa alla rimozione di azoto totale a livello di bacino.
Si segnala inoltre che la concentrazione media di azoto organico in uscita è pari a 1,84 mg/L, valore superiore rispetto a ciò che si rileva abitualmente (circa 1 mg/L). Una possibile spiegazione è legata alle posizioni in cui si introducono i ricircoli provenienti dalla linea fanghi; generalmente il punto di immissione è in testa all'impianto, ma, in questo caso, il surnatante dell'ispessimento è immesso a metà della vasca di ossidazione e il centrato alla fine. Di conseguenza, i ricircoli non subiscono un trattamento biologico completo.
L'impianto lavora su tre linee biologiche parallele e, come dimostrato dalla concentrazione di solidi in vasca di ossidazione, tra di esse è presente uno squilibrio.
Nella vasche di ossidazione, l'ossigeno disciolto dovrebbe mantenersi attorno al valore di concentrazione stabilito come set point, ma l'utilizzo di compressori sovradimensionati fa sì che le concentrazioni siano spesso superiori a tale valore anche a regime di funzionamento minimo dei compressori.

Nel caso dell'impianto B, la portata mediana è di 1.600 m\textsuperscript{3}/d, a fronte di 2.000 m\textsuperscript{3}/d di portata di progetto. La potenzialità autorizzata di 10.000 AE, invece, rappresenta circa il 50° percentile della distribuzione di carico di COD in ingresso.
Il liquame è a media-forte concentrazione con valori di COD compresi tra 500 e 1.500 mg/L e valori medi di azoto totale di 60 mg/L. Inoltre, le sue caratteristiche sono sbilanciate verso il fosforo.
L'effluente è caratterizzato da concentrazioni di inquinanti sempre compatibili con i limiti allo scarico.
Il rendimento del processo di nitrificazione, a differenza di quanto avviene per l'impianto A, non risente dell'influenza negativa delle basse temperature.
Sebbene il ricircolo del surnatante e del centrato della linea fanghi sia introdotto in testa alla denitrificazione, la concentrazione di azoto organico in uscita assume una valore medio di 1,62 mg/L che è maggiore del valore tipico di 1 mg/L.
In vasca di ossidazione si ha una concentrazione di ossigeno molto variabile perché i compressori forniscono aria in eccesso e quindi non si riesce a mantenere il valore di set point.

Queste conclusioni derivano dall'utilizzo parallelo dei metodi convenzionali e di quelli più avanzati.
Nella maggior parte dei casi, l'applicazione di metodi diversi ha portato a risultati coerenti. Ciò che è stato osservato con l'analisi visuale dei dati è quasi sempre stato confermato dalle analisi statistiche più approfondite. Queste ultime, oltre che essere una conferma, hanno permesso di individuare caratteristiche dei dati che non si erano scorte con l'analisi precedente. Inoltre, esse hanno il vantaggio di restituire informazioni più oggettive poiché fornite in termini quantitativi e non qualitativi.

Dopo aver trattato gli \textit{outliers}, distinguendo i valori anomali da modificare o eliminare da quelli da conservare, l'individuazione del trend è stata condotta per mezzo della regressione lineare e del metodo LOESS. Questa analisi ha permesso di quantificare trend già evidenti dall'analisi visuale, come nel caso dell'andamento decrescente della portata e dei carichi in ingresso nell'anno 2015 per l'impianto A e delle concentrazioni in ingresso di BOD\textsubscript{5}, COD, SST e fosforo totale per l'impianto B. Per altri parametri, come concentrazione in ingresso di azoto totale e di fosforo totale dell'impianto A e concentrazione in ingresso di azoto totale dell'impianto B, è stato possibile notare che quello che sembrava essere un trend debolmente decrescente dallo studio della rappresentazione grafica, in realtà è un trend più marcato. Inoltre, in molto casi, soprattutto per quanto riguarda le concentrazioni in uscita degli inquinanti per entrambi gli impianti, la regressione lineare ha evidenziato dei trend che non erano stati notati con l'analisi esplorativa.

I trend individuati sono stati rimossi dalle \textit{time series} al fine di ricercare eventuali periodicità per mezzo della funzione di autocorrelazione. Esso ha generalmente confermato quanto già osservato tramite l'analisi visuale dei dati, come è particolarmente evidente nel caso delle temperature in ingresso e in uscita per entrambi gli impianti. Solamente per l'impianto A, invece, l'autocorrelazione ha evidenziato la presenza di periodicità che non si erano viste con l'analisi visuale. Alcuni esempi sono le concentrazioni in uscita di azoto nitrico e di fosforo totale e le concentrazioni di solidi sospesi nel ricircolo dei fanghi nella linea 3.

Infine, la correlazione, che con l'analisi visuale era stata valutata semplicemente rappresentando in uno stesso grafico più parametri e controllando che essi avessero indicativamente variazioni concordi, è stata quantificata per mezzo del coefficiente di correlazione di Spearman. Benché questo metodo non abbia condotto a risultati diversi da quelli già ottenuti, ha permesso di confermare a livello oggettivo ciò che, invece, poteva dipendere dalla sensibilità dell'osservatore.

In conclusione, si ritiene che l'analisi esplorativa debba essere affiancata e arricchita dai metodi statistici più avanzati.

Numerosi sono gli approcci che possono ancora essere utilizzati per rendere più dettagliato e più significativo lo studio dei dati e ci si auspica che essi vengano approfonditi in lavori futuri.


